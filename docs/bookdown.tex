% Options for packages loaded elsewhere
\PassOptionsToPackage{unicode}{hyperref}
\PassOptionsToPackage{hyphens}{url}
%
\documentclass[
]{book}
\usepackage{lmodern}
\usepackage{amssymb,amsmath}
\usepackage{ifxetex,ifluatex}
\ifnum 0\ifxetex 1\fi\ifluatex 1\fi=0 % if pdftex
  \usepackage[T1]{fontenc}
  \usepackage[utf8]{inputenc}
  \usepackage{textcomp} % provide euro and other symbols
\else % if luatex or xetex
  \usepackage{unicode-math}
  \defaultfontfeatures{Scale=MatchLowercase}
  \defaultfontfeatures[\rmfamily]{Ligatures=TeX,Scale=1}
\fi
% Use upquote if available, for straight quotes in verbatim environments
\IfFileExists{upquote.sty}{\usepackage{upquote}}{}
\IfFileExists{microtype.sty}{% use microtype if available
  \usepackage[]{microtype}
  \UseMicrotypeSet[protrusion]{basicmath} % disable protrusion for tt fonts
}{}
\makeatletter
\@ifundefined{KOMAClassName}{% if non-KOMA class
  \IfFileExists{parskip.sty}{%
    \usepackage{parskip}
  }{% else
    \setlength{\parindent}{0pt}
    \setlength{\parskip}{6pt plus 2pt minus 1pt}}
}{% if KOMA class
  \KOMAoptions{parskip=half}}
\makeatother
\usepackage{xcolor}
\IfFileExists{xurl.sty}{\usepackage{xurl}}{} % add URL line breaks if available
\IfFileExists{bookmark.sty}{\usepackage{bookmark}}{\usepackage{hyperref}}
\hypersetup{
  pdftitle={Tutorial for STA2002},
  pdfauthor={Kun HUANG(SDS)},
  hidelinks,
  pdfcreator={LaTeX via pandoc}}
\urlstyle{same} % disable monospaced font for URLs
\usepackage{color}
\usepackage{fancyvrb}
\newcommand{\VerbBar}{|}
\newcommand{\VERB}{\Verb[commandchars=\\\{\}]}
\DefineVerbatimEnvironment{Highlighting}{Verbatim}{commandchars=\\\{\}}
% Add ',fontsize=\small' for more characters per line
\usepackage{framed}
\definecolor{shadecolor}{RGB}{248,248,248}
\newenvironment{Shaded}{\begin{snugshade}}{\end{snugshade}}
\newcommand{\AlertTok}[1]{\textcolor[rgb]{0.94,0.16,0.16}{#1}}
\newcommand{\AnnotationTok}[1]{\textcolor[rgb]{0.56,0.35,0.01}{\textbf{\textit{#1}}}}
\newcommand{\AttributeTok}[1]{\textcolor[rgb]{0.77,0.63,0.00}{#1}}
\newcommand{\BaseNTok}[1]{\textcolor[rgb]{0.00,0.00,0.81}{#1}}
\newcommand{\BuiltInTok}[1]{#1}
\newcommand{\CharTok}[1]{\textcolor[rgb]{0.31,0.60,0.02}{#1}}
\newcommand{\CommentTok}[1]{\textcolor[rgb]{0.56,0.35,0.01}{\textit{#1}}}
\newcommand{\CommentVarTok}[1]{\textcolor[rgb]{0.56,0.35,0.01}{\textbf{\textit{#1}}}}
\newcommand{\ConstantTok}[1]{\textcolor[rgb]{0.00,0.00,0.00}{#1}}
\newcommand{\ControlFlowTok}[1]{\textcolor[rgb]{0.13,0.29,0.53}{\textbf{#1}}}
\newcommand{\DataTypeTok}[1]{\textcolor[rgb]{0.13,0.29,0.53}{#1}}
\newcommand{\DecValTok}[1]{\textcolor[rgb]{0.00,0.00,0.81}{#1}}
\newcommand{\DocumentationTok}[1]{\textcolor[rgb]{0.56,0.35,0.01}{\textbf{\textit{#1}}}}
\newcommand{\ErrorTok}[1]{\textcolor[rgb]{0.64,0.00,0.00}{\textbf{#1}}}
\newcommand{\ExtensionTok}[1]{#1}
\newcommand{\FloatTok}[1]{\textcolor[rgb]{0.00,0.00,0.81}{#1}}
\newcommand{\FunctionTok}[1]{\textcolor[rgb]{0.00,0.00,0.00}{#1}}
\newcommand{\ImportTok}[1]{#1}
\newcommand{\InformationTok}[1]{\textcolor[rgb]{0.56,0.35,0.01}{\textbf{\textit{#1}}}}
\newcommand{\KeywordTok}[1]{\textcolor[rgb]{0.13,0.29,0.53}{\textbf{#1}}}
\newcommand{\NormalTok}[1]{#1}
\newcommand{\OperatorTok}[1]{\textcolor[rgb]{0.81,0.36,0.00}{\textbf{#1}}}
\newcommand{\OtherTok}[1]{\textcolor[rgb]{0.56,0.35,0.01}{#1}}
\newcommand{\PreprocessorTok}[1]{\textcolor[rgb]{0.56,0.35,0.01}{\textit{#1}}}
\newcommand{\RegionMarkerTok}[1]{#1}
\newcommand{\SpecialCharTok}[1]{\textcolor[rgb]{0.00,0.00,0.00}{#1}}
\newcommand{\SpecialStringTok}[1]{\textcolor[rgb]{0.31,0.60,0.02}{#1}}
\newcommand{\StringTok}[1]{\textcolor[rgb]{0.31,0.60,0.02}{#1}}
\newcommand{\VariableTok}[1]{\textcolor[rgb]{0.00,0.00,0.00}{#1}}
\newcommand{\VerbatimStringTok}[1]{\textcolor[rgb]{0.31,0.60,0.02}{#1}}
\newcommand{\WarningTok}[1]{\textcolor[rgb]{0.56,0.35,0.01}{\textbf{\textit{#1}}}}
\usepackage{longtable,booktabs}
% Correct order of tables after \paragraph or \subparagraph
\usepackage{etoolbox}
\makeatletter
\patchcmd\longtable{\par}{\if@noskipsec\mbox{}\fi\par}{}{}
\makeatother
% Allow footnotes in longtable head/foot
\IfFileExists{footnotehyper.sty}{\usepackage{footnotehyper}}{\usepackage{footnote}}
\makesavenoteenv{longtable}
\usepackage{graphicx,grffile}
\makeatletter
\def\maxwidth{\ifdim\Gin@nat@width>\linewidth\linewidth\else\Gin@nat@width\fi}
\def\maxheight{\ifdim\Gin@nat@height>\textheight\textheight\else\Gin@nat@height\fi}
\makeatother
% Scale images if necessary, so that they will not overflow the page
% margins by default, and it is still possible to overwrite the defaults
% using explicit options in \includegraphics[width, height, ...]{}
\setkeys{Gin}{width=\maxwidth,height=\maxheight,keepaspectratio}
% Set default figure placement to htbp
\makeatletter
\def\fps@figure{htbp}
\makeatother
\setlength{\emergencystretch}{3em} % prevent overfull lines
\providecommand{\tightlist}{%
  \setlength{\itemsep}{0pt}\setlength{\parskip}{0pt}}
\setcounter{secnumdepth}{5}
\usepackage{booktabs}
\usepackage[]{natbib}
\bibliographystyle{apalike}

\title{Tutorial for STA2002}
\author{Kun HUANG(SDS)}
\date{2020-09-04}

\usepackage{amsthm}
\newtheorem{theorem}{Theorem}[chapter]
\newtheorem{lemma}{Lemma}[chapter]
\newtheorem{corollary}{Corollary}[chapter]
\newtheorem{proposition}{Proposition}[chapter]
\newtheorem{conjecture}{Conjecture}[chapter]
\theoremstyle{definition}
\newtheorem{definition}{Definition}[chapter]
\theoremstyle{definition}
\newtheorem{example}{Example}[chapter]
\theoremstyle{definition}
\newtheorem{exercise}{Exercise}[chapter]
\theoremstyle{remark}
\newtheorem*{remark}{Remark}
\newtheorem*{solution}{Solution}
\begin{document}
\maketitle

{
\setcounter{tocdepth}{1}
\tableofcontents
}
\hypertarget{prerequisites}{%
\chapter{Prerequisites}\label{prerequisites}}

\emph{Probability and Statistics I(STA2001)} is the prerequisite, which mainly includes the following contents,

\begin{itemize}
\item
  Some usual distributions, like Binomial, Poisson, Normal, Exponential, Gamma, and Chi-square distributions (\href{https://www.tandfonline.com/doi/pdf/10.1080/07408170590948512?needAccess=true}{Relationships among some univariate
  distributions}\citep{song2005relationships});
\item
  Basic terminologies, e.g.,independence, expectation, variation, correlation (coefficient), Bayes, and etc;
\item
  Large number theorem, like Central Limit Theorem(CLT).
\end{itemize}

\hypertarget{sec:T1}{%
\chapter{Tutorial 1}\label{sec:T1}}

\hypertarget{q1}{%
\section{Q1}\label{q1}}

\begin{itemize}
\item
  Moment-generating function \(M(t)\) of a random variable \(X\) defined in \(D\) that has a density function \(f(x)\).
  \begin{align}
  M(t) = \mathbb{E}(e^{tx}) &= \int_{D} e^{tx}f(x)dx\\
  \mathbb{E}(X^{s}) &= M^{(s)}(0)
  \end{align}
\item
  Relationship between \(\bar X=\frac{1}{n}\sum\limits_{i=1}^n X_i\) and \(S^2=\frac{1}{n-1}\sum\limits_{i=1}^n(X_i-\bar X)^2\), independent.
\item
  How to derive a quantity following \(t\) distribution from a norm population.
  \begin{align}
  T=\frac{\frac{\bar{X}-\mu}{\sigma / \sqrt{n}}}{\sqrt{\frac{(n-1) S^{2}}{\sigma^{2}} /(n-1)}}=\frac{\bar{X}-\mu}{S / \sqrt{n}}
  \end{align}
\item
  The \(t\) distribution is symmetric, i.e., \(t_{q}(n) = -t_{1-q}(n), q\in(0,1)\). For example,
\end{itemize}

\begin{Shaded}
\begin{Highlighting}[]
\KeywordTok{qt}\NormalTok{(}\FloatTok{0.025}\NormalTok{, }\DecValTok{8}\NormalTok{, }\DataTypeTok{lower.tail =}\NormalTok{ F)}
\end{Highlighting}
\end{Shaded}

\begin{verbatim}
## [1] 2.306004
\end{verbatim}

\begin{Shaded}
\begin{Highlighting}[]
\OperatorTok{-}\KeywordTok{qt}\NormalTok{(}\DecValTok{1} \OperatorTok{-}\StringTok{ }\FloatTok{0.025}\NormalTok{, }\DecValTok{8}\NormalTok{, }\DataTypeTok{lower.tail =}\NormalTok{ F)}
\end{Highlighting}
\end{Shaded}

\begin{verbatim}
## [1] 2.306004
\end{verbatim}

\begin{itemize}
\tightlist
\item
  Properties of \(F\) distribution.
\end{itemize}

\hypertarget{q2}{%
\section{Q2}\label{q2}}

\begin{itemize}
\item
  Standardize a norm distribution \(X\in\mathcal{N}(\mu,\sigma^2)\), i.e., \(\frac{X-\mu}{\sigma}\in\mathcal{N}(0,1)\).
\item
  The distribution of \(\bar X\) and \(S^2\).
\end{itemize}

\hypertarget{q3}{%
\section{Q3}\label{q3}}

\begin{itemize}
\tightlist
\item
  Central Limit Theorem(CLT)
\end{itemize}

\begin{theorem}
\protect\hypertarget{thm:clt}{}{\label{thm:clt} }\textbf{(Central Limit Theorem)} Let \(X_1,\ldots,X_n\) be independent, identically
distributed (i.i.d.) random variables with finite expectation \(\mu\), and
positive, finite variance \(\sigma^2\), and set \(S_n=X_1 + X_2 + \cdots + X_n\), \(n \ge 1\). Then
\[
    \frac{\bar{S}_n - n\mu}{\sigma \sqrt{n}}\xrightarrow{L} N(0, 1)
    ~\mathrm{as}~n\rightarrow\infty.
\]
\end{theorem}

\begin{itemize}
\item
  The relationship between Binimial distribution and Poisson distribution
\item
  Aware the power of CLT.
\end{itemize}

  \bibliography{book.bib,packages.bib}

\end{document}
