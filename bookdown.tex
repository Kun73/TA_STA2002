% Options for packages loaded elsewhere
\PassOptionsToPackage{unicode}{hyperref}
\PassOptionsToPackage{hyphens}{url}
%
\documentclass[
]{book}
\usepackage{lmodern}
\usepackage{amssymb,amsmath}
\usepackage{ifxetex,ifluatex}
\ifnum 0\ifxetex 1\fi\ifluatex 1\fi=0 % if pdftex
  \usepackage[T1]{fontenc}
  \usepackage[utf8]{inputenc}
  \usepackage{textcomp} % provide euro and other symbols
\else % if luatex or xetex
  \usepackage{unicode-math}
  \defaultfontfeatures{Scale=MatchLowercase}
  \defaultfontfeatures[\rmfamily]{Ligatures=TeX,Scale=1}
\fi
% Use upquote if available, for straight quotes in verbatim environments
\IfFileExists{upquote.sty}{\usepackage{upquote}}{}
\IfFileExists{microtype.sty}{% use microtype if available
  \usepackage[]{microtype}
  \UseMicrotypeSet[protrusion]{basicmath} % disable protrusion for tt fonts
}{}
\makeatletter
\@ifundefined{KOMAClassName}{% if non-KOMA class
  \IfFileExists{parskip.sty}{%
    \usepackage{parskip}
  }{% else
    \setlength{\parindent}{0pt}
    \setlength{\parskip}{6pt plus 2pt minus 1pt}}
}{% if KOMA class
  \KOMAoptions{parskip=half}}
\makeatother
\usepackage{xcolor}
\IfFileExists{xurl.sty}{\usepackage{xurl}}{} % add URL line breaks if available
\IfFileExists{bookmark.sty}{\usepackage{bookmark}}{\usepackage{hyperref}}
\hypersetup{
  pdftitle={Tutorial for STA2002},
  pdfauthor={Kun HUANG(SDS)},
  hidelinks,
  pdfcreator={LaTeX via pandoc}}
\urlstyle{same} % disable monospaced font for URLs
\usepackage{color}
\usepackage{fancyvrb}
\newcommand{\VerbBar}{|}
\newcommand{\VERB}{\Verb[commandchars=\\\{\}]}
\DefineVerbatimEnvironment{Highlighting}{Verbatim}{commandchars=\\\{\}}
% Add ',fontsize=\small' for more characters per line
\usepackage{framed}
\definecolor{shadecolor}{RGB}{248,248,248}
\newenvironment{Shaded}{\begin{snugshade}}{\end{snugshade}}
\newcommand{\AlertTok}[1]{\textcolor[rgb]{0.94,0.16,0.16}{#1}}
\newcommand{\AnnotationTok}[1]{\textcolor[rgb]{0.56,0.35,0.01}{\textbf{\textit{#1}}}}
\newcommand{\AttributeTok}[1]{\textcolor[rgb]{0.77,0.63,0.00}{#1}}
\newcommand{\BaseNTok}[1]{\textcolor[rgb]{0.00,0.00,0.81}{#1}}
\newcommand{\BuiltInTok}[1]{#1}
\newcommand{\CharTok}[1]{\textcolor[rgb]{0.31,0.60,0.02}{#1}}
\newcommand{\CommentTok}[1]{\textcolor[rgb]{0.56,0.35,0.01}{\textit{#1}}}
\newcommand{\CommentVarTok}[1]{\textcolor[rgb]{0.56,0.35,0.01}{\textbf{\textit{#1}}}}
\newcommand{\ConstantTok}[1]{\textcolor[rgb]{0.00,0.00,0.00}{#1}}
\newcommand{\ControlFlowTok}[1]{\textcolor[rgb]{0.13,0.29,0.53}{\textbf{#1}}}
\newcommand{\DataTypeTok}[1]{\textcolor[rgb]{0.13,0.29,0.53}{#1}}
\newcommand{\DecValTok}[1]{\textcolor[rgb]{0.00,0.00,0.81}{#1}}
\newcommand{\DocumentationTok}[1]{\textcolor[rgb]{0.56,0.35,0.01}{\textbf{\textit{#1}}}}
\newcommand{\ErrorTok}[1]{\textcolor[rgb]{0.64,0.00,0.00}{\textbf{#1}}}
\newcommand{\ExtensionTok}[1]{#1}
\newcommand{\FloatTok}[1]{\textcolor[rgb]{0.00,0.00,0.81}{#1}}
\newcommand{\FunctionTok}[1]{\textcolor[rgb]{0.00,0.00,0.00}{#1}}
\newcommand{\ImportTok}[1]{#1}
\newcommand{\InformationTok}[1]{\textcolor[rgb]{0.56,0.35,0.01}{\textbf{\textit{#1}}}}
\newcommand{\KeywordTok}[1]{\textcolor[rgb]{0.13,0.29,0.53}{\textbf{#1}}}
\newcommand{\NormalTok}[1]{#1}
\newcommand{\OperatorTok}[1]{\textcolor[rgb]{0.81,0.36,0.00}{\textbf{#1}}}
\newcommand{\OtherTok}[1]{\textcolor[rgb]{0.56,0.35,0.01}{#1}}
\newcommand{\PreprocessorTok}[1]{\textcolor[rgb]{0.56,0.35,0.01}{\textit{#1}}}
\newcommand{\RegionMarkerTok}[1]{#1}
\newcommand{\SpecialCharTok}[1]{\textcolor[rgb]{0.00,0.00,0.00}{#1}}
\newcommand{\SpecialStringTok}[1]{\textcolor[rgb]{0.31,0.60,0.02}{#1}}
\newcommand{\StringTok}[1]{\textcolor[rgb]{0.31,0.60,0.02}{#1}}
\newcommand{\VariableTok}[1]{\textcolor[rgb]{0.00,0.00,0.00}{#1}}
\newcommand{\VerbatimStringTok}[1]{\textcolor[rgb]{0.31,0.60,0.02}{#1}}
\newcommand{\WarningTok}[1]{\textcolor[rgb]{0.56,0.35,0.01}{\textbf{\textit{#1}}}}
\usepackage{longtable,booktabs}
% Correct order of tables after \paragraph or \subparagraph
\usepackage{etoolbox}
\makeatletter
\patchcmd\longtable{\par}{\if@noskipsec\mbox{}\fi\par}{}{}
\makeatother
% Allow footnotes in longtable head/foot
\IfFileExists{footnotehyper.sty}{\usepackage{footnotehyper}}{\usepackage{footnote}}
\makesavenoteenv{longtable}
\usepackage{graphicx,grffile}
\makeatletter
\def\maxwidth{\ifdim\Gin@nat@width>\linewidth\linewidth\else\Gin@nat@width\fi}
\def\maxheight{\ifdim\Gin@nat@height>\textheight\textheight\else\Gin@nat@height\fi}
\makeatother
% Scale images if necessary, so that they will not overflow the page
% margins by default, and it is still possible to overwrite the defaults
% using explicit options in \includegraphics[width, height, ...]{}
\setkeys{Gin}{width=\maxwidth,height=\maxheight,keepaspectratio}
% Set default figure placement to htbp
\makeatletter
\def\fps@figure{htbp}
\makeatother
\setlength{\emergencystretch}{3em} % prevent overfull lines
\providecommand{\tightlist}{%
  \setlength{\itemsep}{0pt}\setlength{\parskip}{0pt}}
\setcounter{secnumdepth}{5}

\title{Tutorial for STA2002}
\author{Kun HUANG(SDS)}
\date{2020-11-10}

\usepackage{amsthm}
\newtheorem{theorem}{Theorem}[chapter]
\newtheorem{lemma}{Lemma}[chapter]
\newtheorem{corollary}{Corollary}[chapter]
\newtheorem{proposition}{Proposition}[chapter]
\newtheorem{conjecture}{Conjecture}[chapter]
\theoremstyle{definition}
\newtheorem{definition}{Definition}[chapter]
\theoremstyle{definition}
\newtheorem{example}{Example}[chapter]
\theoremstyle{definition}
\newtheorem{exercise}{Exercise}[chapter]
\theoremstyle{remark}
\newtheorem*{remark}{Remark}
\newtheorem*{solution}{Solution}
\begin{document}
\maketitle

{
\setcounter{tocdepth}{1}
\tableofcontents
}
\hypertarget{prerequisites}{%
\chapter*{Prerequisites}\label{prerequisites}}
\addcontentsline{toc}{chapter}{Prerequisites}

\emph{Probability and Statistics I(STA2001)} is the prerequisite, which mainly includes the following contents,

\begin{itemize}
\item
  Some usual distributions, like Binomial, Poisson, Normal, Exponential, Gamma, and Chi-square distributions (\href{https://www.tandfonline.com/doi/pdf/10.1080/07408170590948512?needAccess=true}{Relationships among some univariate
  distributions}(Song \protect\hyperlink{ref-song2005relationships}{2005}));
\item
  Basic terminologies, e.g.,independence, expectation, variation, correlation (coefficient), Bayes, and etc;
\item
  Large number theorem, like Central Limit Theorem(CLT).
\end{itemize}

\hypertarget{sec:T1}{%
\chapter{Tutorial 1}\label{sec:T1}}

\hypertarget{q1}{%
\section{Q1}\label{q1}}

\begin{itemize}
\item
  Moment-generating function \(M(t)\) of a random variable \(X\) defined in \(D\) that has a density function \(f(x)\).
  \begin{align}
  M(t) = \mathbb{E}(e^{tx}) &= \int_{D} e^{tx}f(x)dx\\
  \mathbb{E}(X^{s}) &= M^{(s)}(0)
  \end{align}
\item
  Relationship between \(\bar X=\frac{1}{n}\sum\limits_{i=1}^n X_i\) and \(S^2=\frac{1}{n-1}\sum\limits_{i=1}^n(X_i-\bar X)^2\), independent.
\item
  How to derive a quantity following \(t\) distribution from a norm population.
  \begin{align}
  T=\frac{\frac{\bar{X}-\mu}{\sigma / \sqrt{n}}}{\sqrt{\frac{(n-1) S^{2}}{\sigma^{2}} /(n-1)}}=\frac{\bar{X}-\mu}{S / \sqrt{n}}
  \end{align}
\item
  The \(t\) distribution is symmetric, i.e., \(t_{q}(n) = -t_{1-q}(n), q\in(0,1)\). For example,
\end{itemize}

\begin{Shaded}
\begin{Highlighting}[]
\KeywordTok{qt}\NormalTok{(}\FloatTok{0.025}\NormalTok{, }\DecValTok{8}\NormalTok{, }\DataTypeTok{lower.tail =}\NormalTok{ F)}
\end{Highlighting}
\end{Shaded}

\begin{verbatim}
## [1] 2.306004
\end{verbatim}

\begin{Shaded}
\begin{Highlighting}[]
\OperatorTok{-}\KeywordTok{qt}\NormalTok{(}\DecValTok{1} \OperatorTok{-}\StringTok{ }\FloatTok{0.025}\NormalTok{, }\DecValTok{8}\NormalTok{, }\DataTypeTok{lower.tail =}\NormalTok{ F)}
\end{Highlighting}
\end{Shaded}

\begin{verbatim}
## [1] 2.306004
\end{verbatim}

\begin{itemize}
\tightlist
\item
  Properties of \(F\) distribution: \(F_{0.95}(9,24)= \frac{1}{F_{0.05}(24,9)}\)
\end{itemize}

\hypertarget{q2}{%
\section{Q2}\label{q2}}

\begin{itemize}
\item
  Standardize a norm distribution \(X\in\mathcal{N}(\mu,\sigma^2)\), i.e., \(\frac{X-\mu}{\sigma}\in\mathcal{N}(0,1)\).
\item
  The distribution of \(\bar X\) and \(S^2\).
\end{itemize}

\hypertarget{q3}{%
\section{Q3}\label{q3}}

\begin{itemize}
\tightlist
\item
  Central Limit Theorem(CLT)
\end{itemize}

\begin{theorem}
\protect\hypertarget{thm:clt}{}{\label{thm:clt} }\textbf{(Central Limit Theorem)} Let \(X_1,\ldots,X_n\) be independent, identically
distributed (i.i.d.) random variables with finite expectation \(\mu\), and
positive, finite variance \(\sigma^2\), and set \(S_n=X_1 + X_2 + \cdots + X_n\), \(n \ge 1\). Then
\[
    \frac{S_n - n\mu}{\sigma \sqrt{n}}\xrightarrow{L} N(0, 1)
    ~\mathrm{as}~n\rightarrow\infty.
\]
\end{theorem}

\begin{itemize}
\item
  The relationship between Binomial distribution \(b(n,p)\) and Poisson distribution \(\mathrm{Pois}(\lambda)\): \(\infty > np = \lambda, n\rightarrow\infty\)
\item
  Aware the power of CLT.
\end{itemize}

\hypertarget{sec:T2}{%
\chapter{Tutorial 2}\label{sec:T2}}

\hypertarget{q1-1}{%
\section{Q1}\label{q1-1}}

\begin{itemize}
\item
  Derive moments from a given pdf \(f(x)\). \(EX = \int xf(x)dx, EX^2=\int x^2f(x)dx\).
\item
  Derive variance from the first and second moments,i.e., \(Var(X)=EX^2-E^2X\).
\item
  \(E(aX+bY+c) = aEX + bEY+c\), \(Var(aX+bY+c) = a^2Var(X)+b^2Var(Y)\). The latter needs \(X\) and \(Y\) are independent.
\item
  CLT approximation.
\end{itemize}

\hypertarget{q2-1}{%
\section{Q2}\label{q2-1}}

\begin{definition}[Poisson Process]
\protect\hypertarget{def:unnamed-chunk-1}{}{\label{def:unnamed-chunk-1} \iffalse (Poisson Process) \fi{} }Let \(N(t)\) be the number of events happens during the time interval \([0,t]\), if \(N(t)\) satisfies the following:

\begin{itemize}
\tightlist
\item
  N(0) = 0;
\item
  has independent increments, and
\item
  \(\forall \tau>0\), \(P(N(t+\tau)-N(t) = n)= \frac{(\lambda \tau)^n}{n!}e^{-\lambda \tau}\)
\end{itemize}

we call \(\{N(t),t\geq0\}\) is a Poisson process with rate \(\lambda\).
\end{definition}

\begin{itemize}
\tightlist
\item
  Let \((W_n>t)\) be the \(n-th\) random event happens after time \(t\), then \(W_n\sim Gamma(n, \lambda)\). In fact, \(Gamma(n,\lambda)\) can be seen as the time to be waited until the \(n-th\) event.
\end{itemize}

\hypertarget{q3-1}{%
\section{Q3}\label{q3-1}}

\begin{proposition}
\protect\hypertarget{prp:unnamed-chunk-2}{}{\label{prp:unnamed-chunk-2} }Suppose the random variable \(X\) has a pdf \(f(x)\), let \(Y = T(X)\), where \(T:\mathbb{R}\rightarrow \mathbb{R}\) is an invertible transformation. Then the pdf \(g(y)\) of \(Y\) is
\begin{equation*}
g(y) = f(T^{-1}(y))\frac{d T^{-1}(y)}{dy}
\end{equation*}
\end{proposition}

For example, suppose \(X\sim\mathcal{N}(\mu,\sigma^2)\), then \(Y=aX+b\sim\mathcal{N}(a\mu+b, (a\sigma)^2)\).

\hypertarget{q4}{%
\section{Q4}\label{q4}}

\begin{theorem}[Chebyshev's Inequality]
\protect\hypertarget{thm:unnamed-chunk-3}{}{\label{thm:unnamed-chunk-3} \iffalse (Chebyshev's Inequality) \fi{} }Let \(X\) be a random variable with finite mean \(\mu\) and variance \(\sigma^2>0\). Then \(\forall k>0\),
\begin{equation*}
P(|X-\mu|\geq k\sigma) \leq \frac{1}{k^2}
\end{equation*}

Additionally, let \(k\sigma = \varepsilon\), the above becomes,
\begin{equation*}
P(|X-\mu|\geq \varepsilon) \leq \frac{\sigma^2}{\varepsilon^2}
\end{equation*}
\end{theorem}

\hypertarget{tutorial-3}{%
\chapter{Tutorial 3}\label{tutorial-3}}

\hypertarget{method-of-moment-estimatormme}{%
\section{Method of moment estimator(MME)}\label{method-of-moment-estimatormme}}

Suppose that the problem is to estimate \(k\) unknown parameters \(\boldsymbol{\theta} := (\theta_1,\theta_2,...,\theta_k)^{T}\) characterizing the distribution \(f_X(x;\boldsymbol{\theta})\) of the random variable \(X\). Suppose the first \(k\) moments of the true distribution can be expressed by the function of \(\boldsymbol{\theta}\), i.e.,
\begin{align}
\mu_{1} & \equiv \mathrm{E}[W]=g_{1}\left(\theta_{1}, \theta_{2}, \ldots, \theta_{k}\right) \\
\mu_{2} & \equiv \mathrm{E}\left[W^{2}\right]=g_{2}\left(\theta_{1}, \theta_{2}, \ldots, \theta_{k}\right) \\
& \vdots \\
\mu_{k} & \equiv \mathrm{E}\left[W^{k}\right]=g_{k}\left(\theta_{1}, \theta_{2}, \ldots, \theta_{k}\right)
\end{align}
Suppose a sample of size \(n\) is drawn, having the values of \(x_1,x_2,...,x_n\), let
\begin{align}
\mu_j = \frac{1}{n}\sum_{i=1}^n x_i^j, j=1,2,...,k
\end{align}
Solve the above \(k\) equations, we derive the method of moment estimator of \(\boldsymbol{\theta}\).

\hypertarget{maximum-likelihood-estimatormle.}{%
\section{Maximum likelihood estimator(MLE).}\label{maximum-likelihood-estimatormle.}}

Suppose we have a sample of size \(n\), \(X_1,...,X_n\) i.i.d drawn from a population distribution \(f_X(x;\boldsymbol{\theta}), \boldsymbol{\theta} = (\theta_1,\theta_2,...,\theta_k)^{T}\). Define the likelihood function to be
\[
L(\boldsymbol\theta) = \prod_{i=1}^n f(x_i;\boldsymbol{\theta})
\]
The log-likelihood function is defined by \(\ell(\boldsymbol\theta)=\log L(\boldsymbol\theta)\). The maximum likelihood estimator \(\hat{\boldsymbol\theta}\) is determined to maximize \(L(\boldsymbol\theta)\), i.e.,
\begin{equation}
\hat{\boldsymbol\theta} =\underset{\boldsymbol\theta\in \Theta}{\max} L(\boldsymbol\theta)
\end{equation}

\hypertarget{confidence-interval}{%
\chapter{Confidence Interval}\label{confidence-interval}}

\begin{definition}[Confidence Interval]
\protect\hypertarget{def:unnamed-chunk-4}{}{\label{def:unnamed-chunk-4} \iffalse (Confidence Interval) \fi{} }Given a sample \(X_1,X_2,...,X_n\) of the population \(X\sim f(x;\theta)\) and \(\alpha\in[0,1]\), a \((1-\alpha)\) confidence interval \(\left(a(X_1,X_2,...,X_n), b(X_1,X_2,...,X_n)\right)\) for the parameter \(\theta\) is defined such that,
\begin{equation}
P\left[a(X_1,X_2,...,X_n)< \theta <b(X_1,X_2,...,X_n)\right] = 1-\alpha
\end{equation}
\end{definition}

\textbf{\href{https://en.wikipedia.org/wiki/Confidence_interval\#Meaning_and_interpretation}{Interpretation and misunderstanding}}

\hypertarget{q1-2}{%
\section{Q1}\label{q1-2}}

\begin{definition}[t-distribution]
\protect\hypertarget{def:unnamed-chunk-5}{}{\label{def:unnamed-chunk-5} \iffalse (t-distribution) \fi{} }Suppose \(X\sim N(0, 1)\), \(U\sim \chi^2(n)\), and \(X\) are independent from \(Y\), then
\(\frac{X}{\sqrt{U/n}}\) has a (student) t distribution with n degrees of freedom, i.e.,
\[
  \frac{X}{\sqrt{U/n}} \sim t(n)
\]
\end{definition}

Confidence Interval of normal population \(X_i\stackrel{i.i.d.}\sim \mathcal{N}(\mu,\sigma^2),i=1,2,...,n\). We have
\begin{align}
\bar X \sim \mathcal{N}(\mu,\frac{\sigma^2}{n}), \quad \frac{(n-1)S^2}{\sigma^2}\sim \chi^2(n-1)
\end{align}
It can be proved that \(\bar X\) and \(S^2\) are independent. Then,
\begin{equation}
\frac{\frac{\bar X - \mu}{\sqrt{\sigma^2/n}}}{\sqrt{\frac{(n-1)S^2}{\sigma^2} / (n-1)}} = \frac{\bar X -\mu}{S/\sqrt{n}}\sim t(n-1)
\label{eq:tn}
\end{equation}
We call such a method the pivotal approach. A pivotal quantity or pivot is a function of observations and unobservable parameters such that the function's probability distribution does not depend on the unknown parameters. For example, \(\frac{\bar X - \mu}{\sqrt{\sigma^2/n}}\sim \mathcal{N}(0,1)\) is a pivot. From \eqref{eq:tn}, we derive the \((1-\alpha)\) confidence interval for the mean \(\mu\) when \(\sigma^2\) is unknown, i.e.,
\begin{equation}
\bar X \pm t_{\alpha/2}(n-1)\frac{S}{\sqrt{n}}
\end{equation}

\begin{Shaded}
\begin{Highlighting}[]
\KeywordTok{qt}\NormalTok{(}\FloatTok{0.05}\OperatorTok{/}\DecValTok{2}\NormalTok{, }\DecValTok{8}\NormalTok{, }\DataTypeTok{lower.tail =}\NormalTok{ F)}
\end{Highlighting}
\end{Shaded}

\begin{verbatim}
## [1] 2.306004
\end{verbatim}

\begin{Shaded}
\begin{Highlighting}[]
\KeywordTok{qnorm}\NormalTok{(}\FloatTok{0.01}\OperatorTok{/}\DecValTok{2}\NormalTok{, }\DataTypeTok{lower.tail =}\NormalTok{ F)}
\end{Highlighting}
\end{Shaded}

\begin{verbatim}
## [1] 2.575829
\end{verbatim}

\hypertarget{q2-2}{%
\section{Q2}\label{q2-2}}

\begin{theorem}[Welch’s t-interval]
\protect\hypertarget{thm:unnamed-chunk-7}{}{\label{thm:unnamed-chunk-7} \iffalse (Welch's t-interval) \fi{} }Let \(X_1,X_2,...,X_n\stackrel{i.i.d.}{\sim}\mathcal{N}(\mu_X,\sigma^2_X)\) and \(Y_1,Y_2,...,Y_m\stackrel{i.i.d.}{\sim}\mathcal{N}(\mu_Y,\sigma^2_Y)\) be independent random variables. Then an approximate \((1-\alpha)\) C.I. for \(\mu_X-\mu_Y\) is
\[
  \bar{X}-\bar{Y} \pm t_{\alpha / 2}(r) \sqrt{\frac{S_{X}^{2}}{n}+\frac{S_{Y}^{2}}{m}}
\]
where
\[
  r=\left\lfloor\frac{\left(\frac{S_{\mathrm{X}}^{2}}{n}+\frac{S_{\mathrm{X}}^{2}}{m}\right)^{2}}{\frac{1}{n-1}\left(\frac{S_{\mathrm{X}}^{2}}{n}\right)^{2}+\frac{1}{m-1}\left(\frac{S_{\mathrm{Y}}^{2}}{m}\right)^{2}}\right\rfloor
\]
\end{theorem}

Let \(X_1,X_2,...,X_n\stackrel{i.i.d.}{\sim}\mathcal{N}(\mu_X,\sigma^2_X)\) and \(Y_1,Y_2,...,Y_m\stackrel{i.i.d.}{\sim}\mathcal{N}(\mu_Y,\sigma^2_Y)\) be independent random variables. We have the following,
\begin{align}
\bar X&\sim \mathcal{N}(\mu_X, \frac{\sigma_X^2}{n}), \quad 
\bar Y \sim \mathcal{N}(\mu_Y, \frac{\sigma_Y^2}{n})\\
\frac{(n-1)S^2_X}{\sigma_X^2} &\sim \chi^2(n-1), \quad \frac{(m-1)S^2_Y}{\sigma_Y^2} \sim \chi^2(m-1)
\end{align}
The two samples are independent, hence,
\begin{equation}
\bar X - \bar Y \sim \mathcal{N}(\mu_X-\mu_Y,\frac{\sigma_X^2}{n}+\frac{\sigma_Y^2}{m})
\end{equation}

\begin{itemize}
\item
  \(\sigma_X=\sigma_Y=\sigma\) and \(\sigma\) is known, then,
  \begin{equation}
  \frac{\bar X - \bar Y - (\mu_X-\mu_Y)}{\sqrt{\frac{\sigma^2}{n}+\frac{\sigma^2}{m}}}\sim \mathcal{N}(0,1)
  \end{equation}
\item
  \(\sigma_X=\sigma_Y=\sigma\) and \(\sigma\) is unknown, then,
  \begin{align}
  \frac{(n-1)S_X^2+(m-1)S_Y^2}{\sigma^2}&\sim \chi^2(n+m-2)\\
  \frac{\bar X - \bar Y-(\mu_X-\mu_Y)/\left(\sqrt{\frac{\sigma^2}{n}+\frac{\sigma^2}{m}}\right)}{\sqrt{\frac{(n-1)S_X^2+(m-1)S_Y^2}{\sigma^2(n+m-2)}}}&\sim t(n+m-2)
  \end{align}
\item
  \(\sigma_X\neq\sigma_Y\) and they are both unknown, use Welch's t-interval or CLT approximation.
\item
  \(m=n\), then \(Z_i = X_i-Y_i\sim\mathcal{N}(\mu_X-\mu_y,\sigma_Z)\) since \((X_i,Y_i)^T\sim \mathcal{N}\left((\mu_X,\mu_Y)^T, \Sigma\right)\). Then the same technique in Q1 can be used.
\end{itemize}

\begin{Shaded}
\begin{Highlighting}[]
\KeywordTok{qt}\NormalTok{(}\FloatTok{0.05} \OperatorTok{/}\StringTok{ }\DecValTok{2}\NormalTok{, }\DecValTok{8}\NormalTok{, }\DataTypeTok{lower.tail =}\NormalTok{ F)}
\end{Highlighting}
\end{Shaded}

\begin{verbatim}
## [1] 2.306004
\end{verbatim}

\hypertarget{q3-2}{%
\section{Q3}\label{q3-2}}

If \(X\sim \chi^2(n)\) and \(Y\sim\chi^2(m)\) are independent, then
\[
\frac{X/n}{Y/m}\sim F(n, m)
\]
Therefore, with samples from two independent normal population, i.e., let \(X_1,X_2,...,X_n\stackrel{i.i.d.}{\sim}\mathcal{N}(\mu_X,\sigma^2_X)\) and \(Y_1,Y_2,...,Y_m\stackrel{i.i.d.}{\sim}\mathcal{N}(\mu_Y,\sigma^2_Y)\) be independent, we have a pivot
\begin{equation}
\frac{\frac{(n-1)S_X^2}{\sigma^2_X} / (n-1)}{\frac{(m-1)S_Y^2}{\sigma^2_Y}/(m-1)} = \frac{S_X^2/\sigma^2_X}{S_Y^2/\sigma^2_Y}\sim F(n-1, m-1)
\end{equation}

\begin{Shaded}
\begin{Highlighting}[]
\NormalTok{alpha <-}\StringTok{ }\FloatTok{0.02}
\KeywordTok{qf}\NormalTok{(alpha }\OperatorTok{/}\StringTok{ }\DecValTok{2}\NormalTok{, }\DecValTok{12}\NormalTok{, }\DecValTok{8}\NormalTok{, }\DataTypeTok{lower.tail =}\NormalTok{ F)}
\end{Highlighting}
\end{Shaded}

\begin{verbatim}
## [1] 5.666719
\end{verbatim}

\begin{Shaded}
\begin{Highlighting}[]
\KeywordTok{qf}\NormalTok{(alpha }\OperatorTok{/}\StringTok{ }\DecValTok{2}\NormalTok{, }\DecValTok{8}\NormalTok{, }\DecValTok{12}\NormalTok{, }\DataTypeTok{lower.tail =}\NormalTok{ F)}
\end{Highlighting}
\end{Shaded}

\begin{verbatim}
## [1] 4.499365
\end{verbatim}

\begin{equation}
F_{1-\alpha / 2}(r_1, r_2) = \frac{1}{F_{\alpha / 2}(r_2, r_1)}
\end{equation}

\hypertarget{q4-1}{%
\section{Q4}\label{q4-1}}

According to the central limit theorem(CLT), we have an approximate pivot
\begin{equation}
\frac{\bar X - EX}{\sqrt{Var X}}\rightarrow \mathcal{N}(0,1)
\end{equation}

\begin{Shaded}
\begin{Highlighting}[]
\KeywordTok{qnorm}\NormalTok{(}\FloatTok{0.05} \OperatorTok{/}\StringTok{ }\DecValTok{2}\NormalTok{, }\DataTypeTok{lower.tail =}\NormalTok{ F)}
\end{Highlighting}
\end{Shaded}

\begin{verbatim}
## [1] 1.959964
\end{verbatim}

\hypertarget{solutions}{%
\section{Solutions}\label{solutions}}

\hypertarget{q1-3}{%
\subsection{Q1}\label{q1-3}}

\begin{Shaded}
\begin{Highlighting}[]
\NormalTok{x <-}\StringTok{ }\KeywordTok{c}\NormalTok{(}\FloatTok{21.5}\NormalTok{, }\FloatTok{18.95}\NormalTok{, }\FloatTok{18.55}\NormalTok{, }\FloatTok{19.4}\NormalTok{, }\FloatTok{19.15}\NormalTok{, }\FloatTok{22.35}\NormalTok{, }\FloatTok{22.9}\NormalTok{, }\FloatTok{22.2}\NormalTok{, }\FloatTok{23.1}\NormalTok{)}
\KeywordTok{t.test}\NormalTok{(x, }\DataTypeTok{conf.level =} \FloatTok{0.95}\NormalTok{)}
\end{Highlighting}
\end{Shaded}

\begin{verbatim}
## 
##  One Sample t-test
## 
## data:  x
## t = 33.738, df = 8, p-value = 6.506e-10
## alternative hypothesis: true mean is not equal to 0
## 95 percent confidence interval:
##  19.47149 22.32851
## sample estimates:
## mean of x 
##      20.9
\end{verbatim}

\begin{Shaded}
\begin{Highlighting}[]
\NormalTok{n <-}\StringTok{ }\KeywordTok{qnorm}\NormalTok{(}\FloatTok{0.1}\OperatorTok{/}\DecValTok{2}\NormalTok{, }\DataTypeTok{lower.tail =}\NormalTok{ F)}\OperatorTok{^}\DecValTok{2} \OperatorTok{*}\StringTok{ }\KeywordTok{var}\NormalTok{(x) }\OperatorTok{/}\StringTok{ }\NormalTok{(}\FloatTok{0.5}\NormalTok{)}\OperatorTok{^}\DecValTok{2}
\KeywordTok{print}\NormalTok{(n)}
\end{Highlighting}
\end{Shaded}

\begin{verbatim}
## [1] 37.37708
\end{verbatim}

\hypertarget{q2-3}{%
\subsection{Q2}\label{q2-3}}

\begin{Shaded}
\begin{Highlighting}[]
\NormalTok{x <-}\StringTok{ }\KeywordTok{c}\NormalTok{(}\DecValTok{1612}\NormalTok{, }\DecValTok{1352}\NormalTok{, }\DecValTok{1456}\NormalTok{, }\DecValTok{1222}\NormalTok{, }\DecValTok{1560}\NormalTok{, }\DecValTok{1456}\NormalTok{, }\DecValTok{1924}\NormalTok{)}
\NormalTok{y <-}\StringTok{ }\KeywordTok{c}\NormalTok{(}\DecValTok{1082}\NormalTok{, }\DecValTok{1300}\NormalTok{, }\DecValTok{1092}\NormalTok{, }\DecValTok{1040}\NormalTok{, }\DecValTok{910}\NormalTok{, }\DecValTok{1248}\NormalTok{, }\DecValTok{1092}\NormalTok{, }\DecValTok{1040}\NormalTok{, }\DecValTok{1092}\NormalTok{, }\DecValTok{1288}\NormalTok{)}
\KeywordTok{t.test}\NormalTok{(x,y, }\DataTypeTok{var.equal =} \OtherTok{FALSE}\NormalTok{, }\DataTypeTok{conf.level =} \FloatTok{0.95}\NormalTok{)}
\end{Highlighting}
\end{Shaded}

\begin{verbatim}
## 
##  Welch Two Sample t-test
## 
## data:  x and y
## t = 4.235, df = 8.5995, p-value = 0.002427
## alternative hypothesis: true difference in means is not equal to 0
## 95 percent confidence interval:
##  181.7191 604.9095
## sample estimates:
## mean of x mean of y 
##  1511.714  1118.400
\end{verbatim}

Note that R use \(t_{\alpha/2}(8.6)\), so the result of C.I. is different from what we use where the df=8 in t distribution. The pdf of t distribution is
\begin{equation}
f(t)=\frac{\Gamma\left(\frac{\nu+1}{2}\right)}{\sqrt{\nu \pi} \Gamma\left(\frac{\nu}{2}\right)}\left(1+\frac{t^{2}}{\nu}\right)^{-\frac{\nu+1}{2}}
\end{equation}
where \(\nu\) is the degree of freedom.

\hypertarget{q3-3}{%
\subsection{Q3}\label{q3-3}}

\begin{Shaded}
\begin{Highlighting}[]
\NormalTok{r1 <-}\StringTok{ }\DecValTok{9} \OperatorTok{-}\StringTok{ }\DecValTok{1}
\NormalTok{r2 <-}\StringTok{ }\DecValTok{13} \OperatorTok{-}\StringTok{ }\DecValTok{1}
\NormalTok{sx <-}\StringTok{ }\FloatTok{128.41} \OperatorTok{/}\StringTok{ }\DecValTok{12}
\NormalTok{sy <-}\StringTok{ }\FloatTok{36.72} \OperatorTok{/}\StringTok{ }\DecValTok{8}
\NormalTok{alpha <-}\StringTok{ }\FloatTok{0.02}
\NormalTok{ci2 <-}\StringTok{ }\NormalTok{sx }\OperatorTok{/}\StringTok{ }\NormalTok{sy }\OperatorTok{*}\StringTok{ }\KeywordTok{c}\NormalTok{(}\KeywordTok{qf}\NormalTok{(}\DecValTok{1} \OperatorTok{-}\StringTok{ }\NormalTok{alpha }\OperatorTok{/}\StringTok{ }\DecValTok{2}\NormalTok{, r1, r2, }\DataTypeTok{lower.tail =}\NormalTok{ F), }\KeywordTok{qf}\NormalTok{(alpha }\OperatorTok{/}\StringTok{ }\DecValTok{2}\NormalTok{, r1, r2, }\DataTypeTok{lower.tail =}\NormalTok{ F))}
\NormalTok{ci <-}\StringTok{ }\KeywordTok{sqrt}\NormalTok{(ci2)}
\KeywordTok{print}\NormalTok{(ci2)}
\end{Highlighting}
\end{Shaded}

\begin{verbatim}
## [1]  0.4114085 10.4895333
\end{verbatim}

\begin{Shaded}
\begin{Highlighting}[]
\KeywordTok{print}\NormalTok{(ci)}
\end{Highlighting}
\end{Shaded}

\begin{verbatim}
## [1] 0.6414113 3.2387549
\end{verbatim}

\hypertarget{q4-2}{%
\subsection{Q4}\label{q4-2}}

\[
\hat{p}_{1} \pm z_{0.05 / 2} \sqrt{\frac{\hat{p}_{1}\left(1-\hat{p}_{1}\right)}{n_{1}}}
\]

\begin{Shaded}
\begin{Highlighting}[]
\NormalTok{n1 <-}\StringTok{ }\DecValTok{194}
\NormalTok{n2 <-}\StringTok{ }\DecValTok{162}
\NormalTok{y1 <-}\StringTok{ }\DecValTok{28} 
\NormalTok{y2 <-}\StringTok{ }\DecValTok{11}
\NormalTok{p1 <-}\StringTok{ }\NormalTok{y1 }\OperatorTok{/}\StringTok{ }\NormalTok{n1}
\NormalTok{s1 <-}\StringTok{ }\KeywordTok{sqrt}\NormalTok{(n1 }\OperatorTok{*}\StringTok{ }\NormalTok{p1 }\OperatorTok{*}\StringTok{ }\NormalTok{(}\DecValTok{1} \OperatorTok{-}\StringTok{ }\NormalTok{p1)) }\OperatorTok{/}\StringTok{ }\NormalTok{n1}
\NormalTok{p1 }\OperatorTok{+}\StringTok{ }\KeywordTok{c}\NormalTok{(}\OperatorTok{-}\DecValTok{1}\NormalTok{, }\DecValTok{1}\NormalTok{) }\OperatorTok{*}\StringTok{ }\KeywordTok{qnorm}\NormalTok{(}\FloatTok{0.05}\OperatorTok{/}\DecValTok{2}\NormalTok{, }\DataTypeTok{lower.tail =}\NormalTok{ F) }\OperatorTok{*}\StringTok{ }\NormalTok{s1}
\end{Highlighting}
\end{Shaded}

\begin{verbatim}
## [1] 0.0948785 0.1937813
\end{verbatim}

\[
z_{\alpha / 2} \sqrt{\frac{\hat{p}_{1}\left(1-\hat{p}_{1}\right)}{n}} = \varepsilon
\]

\begin{Shaded}
\begin{Highlighting}[]
\NormalTok{alpha <-}\StringTok{ }\FloatTok{0.1}
\NormalTok{ep <-}\StringTok{ }\FloatTok{0.04}
\KeywordTok{qnorm}\NormalTok{(alpha }\OperatorTok{/}\StringTok{ }\DecValTok{2}\NormalTok{, }\DataTypeTok{lower.tail =}\NormalTok{ F)}\OperatorTok{^}\DecValTok{2} \OperatorTok{*}\StringTok{ }\NormalTok{p1 }\OperatorTok{*}\StringTok{ }\NormalTok{(}\DecValTok{1} \OperatorTok{-}\StringTok{ }\NormalTok{p1) }\OperatorTok{/}\StringTok{ }\NormalTok{ep}\OperatorTok{^}\DecValTok{2}
\end{Highlighting}
\end{Shaded}

\begin{verbatim}
## [1] 208.8321
\end{verbatim}

\[
\left(\hat{p}_{1}-\hat{p}_{2}\right)-z_{0.05} \sqrt{\frac{\hat{p}_{1}\left(1-\hat{\rho}_{1}\right)}{n_{1}}+\frac{\hat{\rho}_{2}\left(1-\hat{p}_{2}\right)}{n_{2}}}
\]

\begin{Shaded}
\begin{Highlighting}[]
\NormalTok{p2 <-}\StringTok{ }\NormalTok{y2 }\OperatorTok{/}\StringTok{ }\NormalTok{n2}
\NormalTok{p1 }\OperatorTok{-}\StringTok{ }\NormalTok{p2 }\OperatorTok{-}\StringTok{ }\KeywordTok{qnorm}\NormalTok{(}\FloatTok{0.05}\NormalTok{, }\DataTypeTok{lower.tail =}\NormalTok{ F) }\OperatorTok{*}\StringTok{ }\KeywordTok{sqrt}\NormalTok{(p1 }\OperatorTok{*}\StringTok{ }\NormalTok{(}\DecValTok{1} \OperatorTok{-}\StringTok{ }\NormalTok{p1) }\OperatorTok{/}\StringTok{ }\NormalTok{n1 }\OperatorTok{+}\StringTok{ }\NormalTok{p2 }\OperatorTok{*}\StringTok{ }\NormalTok{(}\DecValTok{1} \OperatorTok{-}\StringTok{  }\NormalTok{p2) }\OperatorTok{/}\StringTok{ }\NormalTok{n2)}
\end{Highlighting}
\end{Shaded}

\begin{verbatim}
## [1] 0.02370925
\end{verbatim}

\hypertarget{simple-linear-regression}{%
\chapter{Simple Linear Regression}\label{simple-linear-regression}}

Consider a simple linear regression model,
\begin{equation}
Y = \alpha + \beta(X-\bar X) + \varepsilon,
\label{eq:slr}
\end{equation}
where \(\varepsilon\sim \mathcal{N}(0, \sigma^2)\). Given \(X\) is not random, we have,
\begin{equation}
Y\sim \mathcal{N}(\alpha + \beta(X-\bar X), \sigma^2)
\end{equation}

\hypertarget{fitting-a-simple-linear-regression-model}{%
\section{Fitting a Simple Linear Regression Model}\label{fitting-a-simple-linear-regression-model}}

Suppose we a series of samples \((x_i,y_i), i=1,2,...,n\) and we want to fit a simple linear regression which has the form of \eqref{eq:slr}. Then the fitted \((\hat \alpha,\hat\beta)\) should minimize the residual, i.e.,

\begin{equation}
(\hat \alpha,\hat\beta) = \underset{\alpha,\beta\in \mathbb{R}}{\arg\min}\sum_{i=1}^n(y_i-\alpha-\beta(x_i-\bar x))^2
\label{eq:residual}
\end{equation}

Solving \eqref{eq:residual}, we derive
\begin{equation}
\hat\alpha = \bar y, \hat\beta = \frac{\sum_{i=1}^{n} y_{i}\left(x_{i}-\bar{x}\right)}{\sum_{i=1}^{n}\left(x_{i}-\bar{x}\right)^{2}}
\end{equation}

Noting that \(y_i = \alpha+\beta(x_i-\bar x)+\varepsilon_i\sim\mathcal{N}(\alpha+\beta(x_i-\bar x), \sigma^2)\), we have
\begin{align}
\hat\alpha &= \bar y \sim \mathcal{N}(\alpha, \frac{\sigma^2}{n})\\
\hat\beta &= \frac{\sum_{i=1}^{n} y_{i}\left(x_{i}-\bar{x}\right)}{\sum_{i=1}^{n}\left(x_{i}-\bar{x}\right)^{2}}
\sim \mathcal{N}(\beta, \frac{\sigma^2}{\sum_{i=1}^n(x_i-\bar x)^2})
\end{align}

The MLE for \(\sigma\) is
\begin{equation}
\hat{\sigma^{2}}=\frac{1}{n} \sum_{i=1}^{n}\left[y_{i}-\hat{\alpha}-\hat{\beta}\left(x_{i}-\bar{x}\right)\right]^{2}
\end{equation}

\hypertarget{a-toy-example}{%
\section{A Toy Example}\label{a-toy-example}}

\begin{Shaded}
\begin{Highlighting}[]
\CommentTok{# Simulated data}
\CommentTok{#' y = 4 + 3x + \textbackslash{}epsilon}
\NormalTok{x <-}\StringTok{ }\KeywordTok{runif}\NormalTok{(}\DecValTok{20}\NormalTok{, }\DataTypeTok{min =} \DecValTok{5}\NormalTok{, }\DataTypeTok{max =} \DecValTok{20}\NormalTok{)}
\NormalTok{y <-}\StringTok{ }\DecValTok{4} \OperatorTok{+}\StringTok{ }\DecValTok{3} \OperatorTok{*}\StringTok{ }\NormalTok{x }\OperatorTok{+}\StringTok{ }\KeywordTok{rnorm}\NormalTok{(}\DecValTok{20}\NormalTok{)}
\NormalTok{slr <-}\StringTok{ }\KeywordTok{lm}\NormalTok{(y }\OperatorTok{~}\StringTok{ }\NormalTok{x)}
\KeywordTok{summary}\NormalTok{(slr)}
\end{Highlighting}
\end{Shaded}

\begin{verbatim}
## 
## Call:
## lm(formula = y ~ x)
## 
## Residuals:
##     Min      1Q  Median      3Q     Max 
## -2.1976 -0.5772  0.1449  0.5021  1.6873 
## 
## Coefficients:
##             Estimate Std. Error t value Pr(>|t|)    
## (Intercept)   4.2570     0.5659   7.522 5.82e-07 ***
## x             2.9759     0.0398  74.773  < 2e-16 ***
## ---
## Signif. codes:  0 '***' 0.001 '**' 0.01 '*' 0.05 '.' 0.1 ' ' 1
## 
## Residual standard error: 0.8826 on 18 degrees of freedom
## Multiple R-squared:  0.9968, Adjusted R-squared:  0.9966 
## F-statistic:  5591 on 1 and 18 DF,  p-value: < 2.2e-16
\end{verbatim}

\begin{Shaded}
\begin{Highlighting}[]
\KeywordTok{names}\NormalTok{(slr)}
\end{Highlighting}
\end{Shaded}

\begin{verbatim}
##  [1] "coefficients"  "residuals"     "effects"       "rank"         
##  [5] "fitted.values" "assign"        "qr"            "df.residual"  
##  [9] "xlevels"       "call"          "terms"         "model"
\end{verbatim}

\begin{Shaded}
\begin{Highlighting}[]
\KeywordTok{fitted}\NormalTok{(slr)}
\end{Highlighting}
\end{Shaded}

\begin{verbatim}
##        1        2        3        4        5        6        7        8 
## 54.98414 19.81821 24.29461 40.13087 22.40393 54.39928 39.25507 24.05793 
##        9       10       11       12       13       14       15       16 
## 45.93421 61.99522 59.83909 59.44902 30.87204 22.55700 56.28820 60.24485 
##       17       18       19       20 
## 47.67986 61.51048 40.95514 51.64456
\end{verbatim}

\hypertarget{hypothesis-testing}{%
\chapter{Hypothesis Testing}\label{hypothesis-testing}}

\hypertarget{summary-of-hypothesis-testing-by-normal-population}{%
\section{Summary of Hypothesis Testing by Normal Population}\label{summary-of-hypothesis-testing-by-normal-population}}

Let samples \(X_1,X_2,...,X_n\) draw from a normal population \(\mathcal{N}(\mu_X,\sigma^2_X)\), then,

\begin{longtable}[]{@{}ccl@{}}
\toprule
& Distribution Under \(H_0\) & Critical Region\tabularnewline
\midrule
\endhead
\(H_0:\mu=\mu_0\),\textless br\textgreater{} \(\sigma\) known &
\(\frac{\bar X-\mu_0}{\sigma/\sqrt{n}}\sim\mathcal{N}(0, 1)\) &
\(H_1:\mu>\mu_0,\quad \frac{\bar x-\mu_0}{\sigma/\sqrt{n}}\geq z_{\alpha}\)
\textless br\textgreater{}
\(H_1:\mu<\mu_0,\quad \frac{\bar x-\mu_0}{\sigma/\sqrt{n}}\leq z_{1-\alpha}=-z_{\alpha}\)\textless br\textgreater{}
\(H_1:\mu\neq\mu_0,\quad |\frac{\bar x-\mu_0}{\sigma/\sqrt{n}}|\geq z_{\alpha/2}\)\tabularnewline
\(H_0:\mu=\mu_0\),\textless br\textgreater{} \(\sigma\) unknown &
\(\frac{\sqrt{n}(\bar X-\mu_0)/\sigma}{\sqrt{\frac{(n-1)S_X^2}{\sigma^2}/(n-1)}}\)
\textless br\textgreater{}\(=\frac{\bar X-\mu_0}{S_X/\sqrt{n}}\sim t(n-1)\)
&
\(H_1:\mu>\mu_0,\quad \frac{\bar X-\mu_0}{S_X/\sqrt{n}}\geq t_{\alpha}(n-1)\)\textless br\textgreater{}
\(H_1:\mu<\mu_0,\quad \frac{\bar X-\mu_0}{S_X/\sqrt{n}}\leq t_{1-\alpha}(n-1)=-t_{\alpha}(n-1)\)\textless br\textgreater{}
\(H_1:\mu\neq\mu_0,\quad |\frac{\bar X-\mu_0}{S_X/\sqrt{n}}|\geq t_{\alpha/2}\)\tabularnewline
\(H_0:\sigma^2 = \sigma^2_0\) &
\(\frac{(n-1)S_X^2}{\sigma_0^2}\sim \chi^2(n-1)\) &
\(H_1:\sigma^2>\sigma^2_0,\quad \frac{(n-1)S_X^2}{\sigma^2_0}\geq \chi_{\alpha}(n-1)\)\textless br\textgreater{}
\(H_1:\sigma^2<\sigma^2_0,\quad \frac{(n-1)S_X^2}{\sigma^2_0}\leq \chi_{1-\alpha}(n-1)\)\textless br\textgreater{}\(H_1:\sigma^2\neq\sigma^2_0,\quad \frac{(n-1)S_X^2}{\sigma^2_0}\geq \chi_{\alpha/2}(n-1)\)
\textless br\textgreater{}
\(\text{ or } \frac{(n-1)S_X^2}{\sigma^2_0}\leq \chi_{1-\alpha/2}(n-1)\)\tabularnewline
\bottomrule
\end{longtable}

\begin{longtable}[]{@{}ccl@{}}
\toprule
\begin{minipage}[b]{0.22\columnwidth}\centering
Null Hypothesis\strut
\end{minipage} & \begin{minipage}[b]{0.35\columnwidth}\centering
Distribution Under \(H_0\)\strut
\end{minipage} & \begin{minipage}[b]{0.35\columnwidth}\raggedright
Critical Region\strut
\end{minipage}\tabularnewline
\midrule
\endhead
\begin{minipage}[t]{0.22\columnwidth}\centering
\(H_0:\mu=\mu_0\), \(\sigma\) known\strut
\end{minipage} & \begin{minipage}[t]{0.35\columnwidth}\centering
\(\frac{\bar X-\mu_0}{\sigma/\sqrt{n}}\sim\mathcal{N}(0, 1)\)\strut
\end{minipage} & \begin{minipage}[t]{0.35\columnwidth}\raggedright
\(H_1:\mu>\mu_0,\quad \frac{\bar x-\mu_0}{\sigma/\sqrt{n}}\geq z_{\alpha}\) \(H_1:\mu<\mu_0,\quad \frac{\bar x-\mu_0}{\sigma/\sqrt{n}}\leq z_{1-\alpha}=-z_{\alpha}\) \(H_1:\mu\neq\mu_0,\quad |\frac{\bar x-\mu_0}{\sigma/\sqrt{n}}|\geq z_{\alpha/2}\)\strut
\end{minipage}\tabularnewline
\begin{minipage}[t]{0.22\columnwidth}\centering
\(H_0:\mu=\mu_0\), \(\sigma\) unknown\strut
\end{minipage} & \begin{minipage}[t]{0.35\columnwidth}\centering
\(\frac{\sqrt{n}(\bar X-\mu_0)/\sigma}{\sqrt{\frac{(n-1)S_X^2}{\sigma^2}/(n-1)}}\) \(=\frac{\bar X-\mu_0}{S_X/\sqrt{n}}\sim t(n-1)\)\strut
\end{minipage} & \begin{minipage}[t]{0.35\columnwidth}\raggedright
\(H_1:\mu>\mu_0,\quad \frac{\bar X-\mu_0}{S_X/\sqrt{n}}\geq t_{\alpha}(n-1)\) \(H_1:\mu<\mu_0,\quad \frac{\bar X-\mu_0}{S_X/\sqrt{n}}\leq t_{1-\alpha}(n-1)\) \(H_1:\mu\neq\mu_0,\quad |\frac{\bar X-\mu_0}{S_X/\sqrt{n}}|\geq t_{\alpha/2}\)\strut
\end{minipage}\tabularnewline
\begin{minipage}[t]{0.22\columnwidth}\centering
\(H_0:\sigma^2 = \sigma^2_0\)\strut
\end{minipage} & \begin{minipage}[t]{0.35\columnwidth}\centering
\(\frac{(n-1)S_X^2}{\sigma_0^2}\sim \chi^2(n-1)\)\strut
\end{minipage} & \begin{minipage}[t]{0.35\columnwidth}\raggedright
\(H_1:\sigma^2>\sigma^2_0,\quad \frac{(n-1)S_X^2}{\sigma^2_0}\geq \chi_{\alpha}(n-1)\) \(H_1:\sigma^2<\sigma^2_0,\quad \frac{(n-1)S_X^2}{\sigma^2_0}\leq \chi_{1-\alpha}(n-1)\)\(H_1:\sigma^2\neq\sigma^2_0,\quad \frac{(n-1)S_X^2}{\sigma^2_0}\geq \chi_{\alpha/2}(n-1)\) \(\text{ or } \frac{(n-1)S_X^2}{\sigma^2_0}\leq \chi_{1-\alpha/2}(n-1)\)\strut
\end{minipage}\tabularnewline
\bottomrule
\end{longtable}

Let samples \(X_1,X_2,...,X_n\) draw from a normal population \(\mathcal{N}(\mu_X,\sigma^2_X)\) and \(Y_1,Y_2,...,Y_m\) from another normal population \(\mathcal{N}(\mu_Y,\sigma^2_Y)\).

\begin{longtable}[]{@{}lll@{}}
\toprule
Null Hypothesis & Distribution Under \(H_0\) & Critical
Region\tabularnewline
\midrule
\endhead
\(H_0:\mu_X=\mu_Y\) ,\textless br\textgreater{}
\(\sigma_X=\sigma_Y=\sigma \), known &
\(Z_1=\frac{\bar X-\bar Y}{\sqrt{\frac{\sigma_X^2}{n}+\frac{\sigma^2_Y}{m}}}\sim\mathcal{N}(0,1)\)
&
\(H_1:\mu_X>\mu_Y,\quad Z_1\geq z_{\alpha}\)\textless br\textgreater{}\(H_1:\mu_X<\mu_Y,\quad Z_1\leq z_{1-\alpha}=-z_{\alpha}\)\textless br\textgreater{}\(H_1:\mu_X\neq\mu_Y,\quad |Z_1|\geq z_{\alpha/2}\)\tabularnewline
\(H_0:\mu_X=\mu_Y\),
\textless br\textgreater{}\(\sigma_X=\sigma_Y=\sigma\), unknown &
\(T_1=\frac{(\bar X-\bar Y)/\sqrt{\frac{\sigma^2}{n}+\frac{\sigma^2}{m}}}{\sqrt{\frac{(n-1)S_X^2+(m-1)S_Y^2}{\sigma^2}/(n+m-2)}}\)
\textless br\textgreater{}\(=\frac{\bar X-\bar Y}{S_p\sqrt{\frac{1}{n}+\frac{1}{m}}}\sim t(n+m-2)\)
&
\(H_1:\mu_X>\mu_Y,\quad T_1\geq t_{\alpha}(n+m-2)\)\textless br\textgreater{}\(H_1:\mu_X<\mu_Y,\quad T_1\leq -t_{\alpha}(n+m-2)\)\textless br\textgreater{}\(H_1:\mu_X\neq\mu_Y,\quad |T_1|\geq t_{\alpha/2}(n+m-2)\)\tabularnewline
\(H_0:\mu_X=\mu_Y\),\textless br\textgreater{}\(\sigma_X\neq\sigma_Y\),
unknown &
\(T_2=\frac{\bar X-\bar Y}{\sqrt{\frac{\sigma_X^2}{n}+\frac{\sigma^2_Y}{m}}}\sim t(r)\),
\textless br\textgreater{}\(r=\left\lfloor\frac{\left(\frac{S_{X}^{2}}{n}+\frac{S_{Y}^{2}}{m}\right)^{2}}{\frac{1}{n-1}\left(\frac{S_{X}^{2}}{n}\right)^{2}+\frac{1}{m-1}\left(\frac{S_{Y}^{2}}{m}\right)^{2}} \right\rfloor\)
&
\(H_1:\mu_X>\mu_Y,\quad T_2\geq t_{\alpha}(r)\)\textless br\textgreater{}\(H_1:\mu_X<\mu_Y,\quad T_2\leq -t_{\alpha}(r)\)\textless br\textgreater{}\(H_1:\mu_X\neq\mu_Y,\quad |T_2|\geq t_{\alpha/2}(r)\)\tabularnewline
\(H_0:\mu_X=\mu_Y\),\textless br\textgreater{}\(m=n\) & \(D_i:=X_i-Y_i\)
\textless br\textgreater{}\(\sim \mathcal{N}(\mu_X-\mu_Y,\sigma^2_Z)\)
\textless br\textgreater{}transform it into the one sample situation
with \(\sigma^2_Z\) unknown &\tabularnewline
\(H_0:\sigma_X^2=\sigma_Y^2\) &
\(F = \frac{\frac{(n-1)S_X^2}{\sigma_X^2}/(n-1)}{\frac{(m-1)S_Y^2}{\sigma_Y^2}/(m-1)}=\frac{S_X^2}{S_Y^2}\)\textless br\textgreater{}\(\sim F(n-1,m-1)\)
&
\(H_1:\sigma_X^2>\sigma_Y^2,\quad F\geq F_{\alpha}(n-1,m-1)\)\textless br\textgreater{}\(H_1:\sigma_X^2<\sigma_Y^2,\quad F\leq F_{1-\alpha}(n-1,m-1)\)\textless br\textgreater{}\(H_1:\sigma_X^\neq\sigma_Y^2,\quad F\geq F_{\alpha/2}(n-1,m-1)\)\textless br\textgreater{}
or \(F\leq F_{1-\alpha/2}(n-1,m-1)\)\tabularnewline
\bottomrule
\end{longtable}

\begin{longtable}[]{@{}lll@{}}
\toprule
\begin{minipage}[b]{0.30\columnwidth}\raggedright
Null Hypothesis\strut
\end{minipage} & \begin{minipage}[b]{0.31\columnwidth}\raggedright
Distribution Under \(H_0\)\strut
\end{minipage} & \begin{minipage}[b]{0.31\columnwidth}\raggedright
Critical Region\strut
\end{minipage}\tabularnewline
\midrule
\endhead
\begin{minipage}[t]{0.30\columnwidth}\raggedright
\(H_0:\mu_X=\mu_Y\) , \(\sigma_X=\sigma_Y=\sigma\), known\strut
\end{minipage} & \begin{minipage}[t]{0.31\columnwidth}\raggedright
\(Z_1=\frac{\bar X-\bar Y}{\sqrt{\frac{\sigma_X^2}{n}+\frac{\sigma^2_Y}{m}}}\sim\mathcal{N}(0,1)\)\strut
\end{minipage} & \begin{minipage}[t]{0.31\columnwidth}\raggedright
\(H_1:\mu_X>\mu_Y,\quad Z_1\geq z_{\alpha}\)\(H_1:\mu_X<\mu_Y,\quad Z_1\leq z_{1-\alpha}=-z_{\alpha}\)\(H_1:\mu_X\neq\mu_Y,\quad |Z_1|\geq z_{\alpha/2}\)\strut
\end{minipage}\tabularnewline
\begin{minipage}[t]{0.30\columnwidth}\raggedright
\(H_0:\mu_X=\mu_Y\), \(\sigma_X=\sigma_Y=\sigma\), unknown\strut
\end{minipage} & \begin{minipage}[t]{0.31\columnwidth}\raggedright
\(T_1=\frac{(\bar X-\bar Y)/\sqrt{\frac{\sigma^2}{n}+\frac{\sigma^2}{m}}}{\sqrt{\frac{(n-1)S_X^2+(m-1)S_Y^2}{\sigma^2}/(n+m-2)}}\) \(=\frac{\bar X-\bar Y}{S_p\sqrt{\frac{1}{n}+\frac{1}{m}}}\sim t(n+m-2)\)\strut
\end{minipage} & \begin{minipage}[t]{0.31\columnwidth}\raggedright
\(H_1:\mu_X>\mu_Y,\quad T_1\geq t_{\alpha}(n+m-2)\)\(H_1:\mu_X<\mu_Y,\quad T_1\leq -t_{\alpha}(n+m-2)\)\(H_1:\mu_X\neq\mu_Y,\quad |T_1|\geq t_{\alpha/2}(n+m-2)\)\strut
\end{minipage}\tabularnewline
\begin{minipage}[t]{0.30\columnwidth}\raggedright
\(H_0:\mu_X=\mu_Y\),\(\sigma_X\neq\sigma_Y\), unknown\strut
\end{minipage} & \begin{minipage}[t]{0.31\columnwidth}\raggedright
\(T_2=\frac{\bar X-\bar Y}{\sqrt{\frac{\sigma_X^2}{n}+\frac{\sigma^2_Y}{m}}}\sim t(r)\), \(r=\left\lfloor\frac{\left(\frac{S_{X}^{2}}{n}+\frac{S_{Y}^{2}}{m}\right)^{2}}{\frac{1}{n-1}\left(\frac{S_{X}^{2}}{n}\right)^{2}+\frac{1}{m-1}\left(\frac{S_{Y}^{2}}{m}\right)^{2}} \right\rfloor\)\strut
\end{minipage} & \begin{minipage}[t]{0.31\columnwidth}\raggedright
\(H_1:\mu_X>\mu_Y,\quad T_2\geq t_{\alpha}(r)\)\(H_1:\mu_X<\mu_Y,\quad T_2\leq -t_{\alpha}(r)\)\(H_1:\mu_X\neq\mu_Y,\quad |T_2|\geq t_{\alpha/2}(r)\)\strut
\end{minipage}\tabularnewline
\begin{minipage}[t]{0.30\columnwidth}\raggedright
\(H_0:\mu_X=\mu_Y\),\(m=n\)\strut
\end{minipage} & \begin{minipage}[t]{0.31\columnwidth}\raggedright
\(D_i:=X_i-Y_i\) \(\sim \mathcal{N}(\mu_X-\mu_Y,\sigma^2_Z)\) transform it into the one sample situation with \(\sigma^2_Z\) unknown\strut
\end{minipage} & \begin{minipage}[t]{0.31\columnwidth}\raggedright
\strut
\end{minipage}\tabularnewline
\begin{minipage}[t]{0.30\columnwidth}\raggedright
\(H_0:\sigma_X^2=\sigma_Y^2\)\strut
\end{minipage} & \begin{minipage}[t]{0.31\columnwidth}\raggedright
\(F = \frac{\frac{(n-1)S_X^2}{\sigma_X^2}/(n-1)}{\frac{(m-1)S_Y^2}{\sigma_Y^2}/(m-1)}=\frac{S_X^2}{S_Y^2}\)\(\sim F(n-1,m-1)\)\strut
\end{minipage} & \begin{minipage}[t]{0.31\columnwidth}\raggedright
\(H_1:\sigma_X^2>\sigma_Y^2,\quad F\geq F_{\alpha}(n-1,m-1)\)\(H_1:\sigma_X^2<\sigma_Y^2,\quad F\leq F_{1-\alpha}(n-1,m-1)\)\(H_1:\sigma_X^\neq\sigma_Y^2,\quad F\geq F_{\alpha/2}(n-1,m-1)\) or \(F\leq F_{1-\alpha/2}(n-1,m-1)\)\strut
\end{minipage}\tabularnewline
\bottomrule
\end{longtable}

We next consider the situation of testing proportion. Let \(X_i\stackrel{i.i.d.}{\sim} Bernoulli(p_X)\) drawn from a specific event and \(Y_i\stackrel{i.i.d.}{\sim} Bernoulli(p_Y)\). We want to infer \(p_X\) and the relationship between \(p_X\) and \(p_Y\). Let \(Z=\sum\limits_{i=1}^n X_i\sim Bin(n,p)\). \(\hat p:=\frac{Z}{n}\) is an unbiased estimator for \(p\). According to the central limit theorem(CLT), we have,
\begin{equation}
\hat p \rightarrow \mathcal{N}(p,\frac{p(1-p)}{n}), n\rightarrow \infty
\end{equation}

Under \(H_0:p_X=p_Y=p\), \(\hat p_{XY}:=\frac{\sum\limits_{i=1}^n X_i+\sum\limits_{i=1}^m Y_i}{n+m}\) is an unbiased estimator of \(p\). Since,
\begin{equation}
\mathbb{E}(\hat p_{XY}) = \frac{\sum\limits_{i=1}^n \mathbb{E} X_i+\sum\limits_{i=1}^m \mathbb{E} Y_i}{n+m}=\frac{np + mp}{m+n}=p
\end{equation}

\begin{longtable}[]{@{}lll@{}}
\toprule
Null Hypothesis & Distribution Under \(H_0\) & Criticall
Region\tabularnewline
\midrule
\endhead
\(H_0:p=p_0\) &
\(Z_p=\frac{\hat p-p_0}{\sqrt{\frac{p_0(1-p_0)}{n}}}\stackrel{approx}{\sim}N(0,1)\)
&
\(H_1:p>p_0,\quad Z_p\geq z_{\alpha}\)\textless br\textgreater{}\(H_1:p<p_0,\quad Z_p\leq z_{1-\alpha}\)\textless br\textgreater{}\(H_1:p\neq p_0,\quad |Z_p|\geq z_{\alpha/2}\)\tabularnewline
\(H_0:p_X=p_Y\) &
\(Z_{XY} = \frac{\hat p_X-\hat p_Y}{\sqrt{\frac{\hat p_{XY}(1-\hat p_{XY})}{n} + \frac{\hat p_{XY}(1-\hat p_{XY})}{m}}}\)
\textless br\textgreater{}\(\stackrel{apprrox}{\sim} \mathcal{N}(0,1),\hat p_{XY}:=\frac{\sum\limits_{i=1}^n X_i+\sum\limits_{i=1}^m Y_i}{n+m}\)
&
\(H_1:p_X>p_Y,\quad Z_{XY}\geq z_{\alpha}\)\textless br\textgreater{}\(H_1:p_X<p_Y,\quad Z_{XY}\leq z_{1-\alpha}=-z_{\alpha}\)\textless br\textgreater{}\(H_1:p_X\neq p_Y,\quad |Z_{XY}|\geq z_{\alpha/2}\)\tabularnewline
\bottomrule
\end{longtable}

\begin{longtable}[]{@{}lll@{}}
\toprule
\begin{minipage}[b]{0.10\columnwidth}\raggedright
Null Hypothesis\strut
\end{minipage} & \begin{minipage}[b]{0.41\columnwidth}\raggedright
Distribution Under \(H_0\)\strut
\end{minipage} & \begin{minipage}[b]{0.41\columnwidth}\raggedright
Criticall Region\strut
\end{minipage}\tabularnewline
\midrule
\endhead
\begin{minipage}[t]{0.10\columnwidth}\raggedright
\(H_0:p=p_0\)\strut
\end{minipage} & \begin{minipage}[t]{0.41\columnwidth}\raggedright
\(Z_p=\frac{\hat p-p_0}{\sqrt{\frac{p_0(1-p_0)}{n}}}\stackrel{approx}{\sim}N(0,1)\)\strut
\end{minipage} & \begin{minipage}[t]{0.41\columnwidth}\raggedright
\(H_1:p>p_0,\quad Z_p\geq z_{\alpha}\)\(H_1:p<p_0,\quad Z_p\leq z_{1-\alpha}\)\(H_1:p\neq p_0,\quad |Z_p|\geq z_{\alpha/2}\)\strut
\end{minipage}\tabularnewline
\begin{minipage}[t]{0.10\columnwidth}\raggedright
\(H_0:p_X=p_Y\)\strut
\end{minipage} & \begin{minipage}[t]{0.41\columnwidth}\raggedright
\(Z_{XY} = \frac{\hat p_X-\hat p_Y}{\sqrt{\frac{\hat p_{XY}(1-\hat p_{XY})}{n} + \frac{\hat p_{XY}(1-\hat p_{XY})}{m}}}\) \(\stackrel{apprrox}{\sim} \mathcal{N}(0,1),\hat p_{XY}:=\frac{\sum\limits_{i=1}^n X_i+\sum\limits_{i=1}^m Y_i}{n+m}\)\strut
\end{minipage} & \begin{minipage}[t]{0.41\columnwidth}\raggedright
\(H_1:p_X>p_Y,\quad Z_{XY}\geq z_{\alpha}\)\(H_1:p_X<p_Y,\quad Z_{XY}\leq z_{1-\alpha}\)\(H_1:p_X\neq p_Y,\quad |Z_{XY}|\geq z_{\alpha/2}\)\strut
\end{minipage}\tabularnewline
\bottomrule
\end{longtable}

\hypertarget{exercise}{%
\section{Exercise}\label{exercise}}

\begin{exercise}
\protect\hypertarget{exr:q71}{}{\label{exr:q71} }To measure air pollution in a home, let \(X\) and \(Y\) equal the amount of suspended particulate matter (in \(\mathrm{g} / \mathrm{m} 3\) ) measured during a 24-hour period in a home in which there is no smoker and a home in which there is a smoker, respectively. We shall test the null hypothesis \(H_{0}: \sigma_{X}^{2} / \sigma_{Y}^{2}=1\) against the one-sided alternative hypothesis \(H_{1}: \sigma_{X}^{2} / \sigma_{Y}^{2}>1\). Suppose both samples are drawn from normal distribution.

\begin{enumerate}
\def\labelenumi{\arabic{enumi}.}
\item
  If a random sample of size \(n=9\) yielded \(\bar{x}=93\) and \(S_{x}=12.9\) while a random sample of size \(m=11\) yielded \(y=132\) and \(S_{y}=7.1,\) define a critical region and give your conclusion if \(\alpha=0.05\).
\item
  Now test \(H_{0}: \mu_{X}=\mu_{Y}\) against \(H_{1}: \mu_{X}<\mu_{Y}\) if \(\alpha=0.05\). \(t_{0.05}(11)=1.796\)
\end{enumerate}
\end{exercise}

Solutions:\\

\begin{enumerate}
\def\labelenumi{\arabic{enumi}.}
\item
  To test \(H_0:\sigma_X^2=\sigma_Y^2\) against \(H_1:\sigma^2_X>\sigma_Y^2\) under normal populations.\\
  \begin{equation}
  F=\frac{S_{x}^{2}}{S_{y}^{2}}=\frac{12.9^{2}}{7.1^{2}}=3.30>3.07=F_{0.05}(8,10)
  \end{equation}
  So we reject \(H_0\) and conclude that \(\sigma_X^2\neq\sigma_Y^2\).
\item
  To test \(H_0:\mu_X=\mu_Y\) against \(H_1:\mu_X<\mu_Y\) under normal populations with variance not being equal.
\end{enumerate}

\begin{equation}
r=\left[\frac{\left(\frac{s_{1}^{2}}{n_{1}}+\frac{s_{2}^{2}}{n_{2}}\right)^{2}}{\frac{\left(s_{1}^{2} / n_{1}\right)^{2}}{n_{1}-1}+\frac{\left(s_{2}^{2} / n_{2}\right)^{2}}{n_{2}-1}}\right]=\left[\frac{\left(\frac{12.9^{2}}{9}+\frac{7.1^{2}}{13}\right)^{2}}{\frac{\left(12.9^{2} / 9\right)^{2}}{9-1}+\frac{\left(7.1^{2} / 11\right)^{2}}{11-1}}\right]=11, \quad t_{1-0.05}(11)=-t_{0.05}(11)=-1.796
\end{equation}

\begin{equation}
t=\frac{\bar{x}_{1}-\bar{y}_{2}}{\sqrt{\frac{S_{X}^{2}}{n}+\frac{S_{Y}^{2}}{m}}}=\frac{93-132}{\sqrt{\frac{12.9^{2}}{9}+\frac{7.1^{2}}{11}}} \approx-8.119<t_{0.95}=-1.796 \Rightarrow \text { Reject } H_{0}
\end{equation}

\begin{exercise}
\protect\hypertarget{exr:q72}{}{\label{exr:q72} }Let \(Y\) be \(b(192, p) .\) We reject \(H_{0}: p=0.75\) and accept \(H_{1}: p>0.75\) if and only if \(Y \geq 152 .\) Use the normal approximation to determine

\begin{enumerate}
\def\labelenumi{\arabic{enumi}.}
\item
  \(\alpha=P(Y \geq 152 ; p=0.75)\).
\item
  \(\beta=P(Y<152)\) when \(p=0.80\).
\end{enumerate}
\end{exercise}

Solution:\\

Proportion for one sample. \(n=192\)

\begin{enumerate}
\def\labelenumi{\arabic{enumi}.}
\item
  \(\sum_{i=1}^n X_i = 152\), according to CLT and half-unit correction
  \begin{align}
  z&=\frac{x-n p}{\sqrt{n p(1-p)}}=\frac{151.5-192(0.75)}{\sqrt{192(0.75)(1-0.75)}} \approx 1.25, z\stackrel{approx}{\sim}\mathcal{N}(0,1)\\
  \alpha&=P(Y \geq 152 ; p=0.75)=P(Y>151.5)=P(z>1.25)=0.1056
  \end{align}
\item
  \(p=0.8\) now, similarly,
\end{enumerate}

\begin{align}
z &=\frac{x-n p}{\sqrt{n p(1-p)}}=\frac{151.5-192(0.80)}{\sqrt{192(0.8)(1-0.8)}} \approx-0.38 \\
\beta &=P(Y<152)=P(Y<151.5)=P(z<-0.38)=P(z>0.38)=0.3520
\end{align}

\begin{exercise}
\protect\hypertarget{exr:q73}{}{\label{exr:q73} }Let \(p\) equal the proportion of drivers who use a seat belt in a state that does not have a mandatory seat belt law. It was claimed that \(p=0.14\) An advertising campaign was conducted to increase this proportion. Two months after the campaign, \(y=104\) out of a random sample of \(n=590\) drivers were wearing their seat belts. Was the campaign successful?

\begin{enumerate}
\def\labelenumi{\arabic{enumi}.}
\item
  Define the null and alternative hypotheses.
\item
  Define a critical region with an \(\alpha=0.01\) significance level. \(z_{0.01} = 2.326\)
\item
  What is your conclusion?
\end{enumerate}
\end{exercise}

Solution:

\begin{enumerate}
\def\labelenumi{\arabic{enumi}.}
\item
  \(H_{0}: p=0.14 \quad \text { against } \quad H_{1}: p>0.14\)
\item
  One sided proportion problem, \(z_{0.01} = 2.326\).
\end{enumerate}

\begin{equation}
C=\{z: z \geq 2.326\} \quad \text { where } \quad z=\frac{y / n-0.14}{\sqrt{(0.14)(0.86) / n}}
\end{equation}

\begin{enumerate}
\def\labelenumi{\arabic{enumi}.}
\setcounter{enumi}{2}
\tightlist
\item
  For this problem, \(y=104, n=590\), the value of test statistics is,
  \begin{equation}
  z=\frac{104 / 590-0.14}{\sqrt{(0.14)(0.86) / 590}}=2.539>2.326
  \end{equation}
\end{enumerate}

Hence, we reject \(H_0\) and conclude that the advertising campaign indeed increases this proportion.

\begin{exercise}
\protect\hypertarget{exr:q74}{}{\label{exr:q74} } For developing countries in Africa and the Americas, let \(p_{1}\) and \(p_{2}\) be the respective proportions of babies with a low birth weight (below 2500 grams). We shall test \(H_{0}: p_{1}=p_{2}\) against the alternative hypothesis \(H_{1}: p_{1}>p_{2}\)

\begin{enumerate}
\def\labelenumi{\arabic{enumi}.}
\item
  Define a critical region that has an \(\alpha=0.05\) significance level. \(z_{0.05}=1.645\)
\item
  If respective random samples of sizes \(n_{1}=900\) and \(n_{2}=700\) yielded \(y_{1}=135\) and \(y_{2}=77\) babies with a low birth weight, what is your conclusion?
\item
  What would your decision be with a significance level of \(\alpha=0.01 ?\) \(z_{0.01}=2.326\)
\item
  What is the \(p\)-value of your test?
\end{enumerate}
\end{exercise}

Solution:

Two samples proportion problem with \(H_0:p_1=p_2\) against \(H_1:p_1>p_2\).

\begin{enumerate}
\def\labelenumi{\arabic{enumi}.}
\item
  \begin{align}
  C=\{z=\frac{\hat{p}_{1}-\hat{p}_{2}}{\sqrt{\hat{p}(1-\hat{p})\left(1 / n_{1}+1 / n_{2}\right)}} \geq 1.645\}
  \end{align}
  where \(\hat p_1 = y_1/n_1,\hat p_2 = y_2/n_2\), and \(\hat p = \frac{y_1+y_2}{n_1+n_2}\).
\item
  Calculate the test statistic,
\end{enumerate}

\begin{align}
z=\frac{0.15-0.11}{\sqrt{(0.1325)(0.8675)(1 / 900+1 / 700)}}=2.341>1.645
\end{align}
Hence, we reject \(H_0\) and conclude that the proportions of babies with a low birth weight in Africa is larger than that in Americas.

\begin{enumerate}
\def\labelenumi{\arabic{enumi}.}
\setcounter{enumi}{2}
\item
  Since \(z=2.341>2.326=z_{0.01}\), we reject \(H_0\) and conclude that the proportions of babies with a low birth weight in Africa is larger than that in Americas.
\item
  The p-value is
\end{enumerate}

\begin{equation}
P(z \geq 2.341)=0.0096
\end{equation}
where \(z\) asymptotically follows \(\mathcal{N}(0,1)\).

\hypertarget{refs}{}
\leavevmode\hypertarget{ref-song2005relationships}{}%
Song, Wheyming Tina. 2005. ``Relationships Among Some Univariate Distributions.'' \emph{IIE Transactions} 37 (7): 651--56.

\end{document}
